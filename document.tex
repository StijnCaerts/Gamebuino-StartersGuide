\documentclass[a4paper,titlepage,12pt]{article}
\usepackage[dutch]{babel}
\usepackage[utf8]{inputenc}
\usepackage{graphicx}
\usepackage{hyperref}
\usepackage{url}
\usepackage{listings}
\usepackage{color}
\usepackage{textcomp}
\usepackage{tikz}
\usepackage{titlepic}
\usepackage{float}

\renewcommand{\familydefault}{\sfdefault}
\renewcommand{\lstlistingname}{Code}


\definecolor{listinggray}{gray}{0.9}
\definecolor{lbcolor}{rgb}{0.9,0.9,0.9}
\definecolor{darkgreen}{rgb}{0.0745,0.568,0.039}
\lstset{
	backgroundcolor=\color{lbcolor},
	tabsize=4,    
	%   rulecolor=,
	language=[GNU]C++,
	basicstyle=\scriptsize,
	upquote=true,
	aboveskip={1.5\baselineskip},
	columns=fixed,
	showstringspaces=false,
	extendedchars=false,
	breaklines=true,
	prebreak = \raisebox{0ex}[0ex][0ex]{\ensuremath{\hookleftarrow}},
	frame=single,
	numbers=left,
	showtabs=false,
	showspaces=false,
	showstringspaces=false,
	identifierstyle=\ttfamily,
	keywordstyle=\color[rgb]{0,0,1},
	commentstyle=\color[rgb]{0.026,0.112,0.095},
	stringstyle=\color[rgb]{0.627,0.126,0.941},
	numberstyle=\color[rgb]{0.205, 0.142, 0.73},
	%        \lstdefinestyle{C++}{language=C++,style=numbers}’.
}
\lstset{
	backgroundcolor=\color{lbcolor},
	tabsize=4,
	language=C++,
	captionpos=b,
	tabsize=3,
	frame=lines,
	numbers=left,
	numberstyle=\tiny,
	numbersep=5pt,
	breaklines=true,
	showstringspaces=false,
	basicstyle=\footnotesize,
	%  identifierstyle=\color{magenta},
	keywordstyle=\color[rgb]{0,0,1},
	commentstyle=\color{darkgreen},
	stringstyle=\color{red}
}

\title{Gamebuino\\ \small{Starters Guide}}
\author{Stijn Caerts}
\titlepic{\includegraphics[width=\linewidth]{assets/img/gamebuino-console.jpg}}

\begin{document}
	\maketitle
	\tableofcontents
	\newpage
	
	\section{Basis}
	Net zoals elk programma dat voor Arduino geschreven is, bevat een programma voor Gamebuino de twee functies \texttt{setup()} en \texttt{loop()}. In deze sectie gaan we dieper in op de basisstructuur die in elk programma aanwezig is.
	
	\begin{lstlisting}[language=C++, caption=Structuur van een Gamebuino programma]
	#include <SPI.h>
	#include <Gamebuino.h>
	
	Gamebuino gb;
	
	void setup(){
		gb.begin();
		gb.titleScreen();
		gb.pickRandomSeed();
	}
	
	void loop(){
		if(gb.update()){
			if (gb.buttons.pressed(BTN_C)) {
				gb.titleScreen();
			}
			// GAME
		}
	}
	\end{lstlisting}
	
	
	\subsection{Libraries importeren}
	Voor we aan de slag kunnen gaan met programmeren, moeten we twee libraries importeren. Deze zullen het ons heel wat makkelijker maken om programma's te schrijven voor Gamebuino. De \emph{SPI} library is vereist voor de communicatie met het scherm van de Gamebuino. Verder gebruiken we de \emph{Gamebuino} library, die voor een handige interface zorgt zodat we gemakkelijk alle functies van het apparaat kunnen aanspreken zodat we ons enkel moeten focussen op het ontwerpen en programmeren van het spel zelf.
	Zie wel dat de libraries die je wilt importeren ook geïnstalleerd zijn in de Arduino IDE, anders zal de library niet gevonden worden en zal je de code niet kunnen compilen.
	\begin{lstlisting}[language=C++, caption={Importeren van libraries}]
	#include <SPI.h>
	#include <Gamebuino.h>
	\end{lstlisting}
	
	
	\subsection{Het \texttt{Gamebuino} object}
	Het \texttt{Gamebuino} object heeft een belangrijke plaats in een game. Je zal dit object vaak gebruiken om functies op te roepen die specifiek voor de Gamebuino zijn. De klasse \texttt{Gamebuino} is onderdeel van de \emph{Gamebuino} library.
	\begin{lstlisting}[language=C++, caption={Aanmaken van een \texttt{Gamebuino} object}]
	Gamebuino gb;
	\end{lstlisting}
	
	
	\subsection{\texttt{setup()}}
	De \texttt{setup()} functie wordt eenmaal uitgevoerd bij het opstarten van het programma. Bij een programma voor Gamebuino bevat deze functie steeds de volgende instructies: \texttt{gb.begin()}, \texttt{gb.titleScreen()} en \texttt{gb.pickRandomSeed()}.
	\begin{lstlisting}[language=C++, caption={\texttt{setup()}}]
	void setup(){
		gb.begin();
		gb.titleScreen();
		gb.pickRandomSeed();
	}
	\end{lstlisting}
	
	
	\subsection{\texttt{loop()}}
	De functie \texttt{loop()} bevat de instructies van de game die je gaat programmeren. Deze functie wordt telkens opnieuw uitgevoerd. Om er voor te zorgen dat we ons programma met vaste intervallen uitvoeren, plaatsen we hier een \texttt{if}-statement met \texttt{gb.update()} waarin we alle code van ons programma plaatsen (zoals te zien is in codefragment \ref{code:loop}).
	
	We moeten in ons programma ook steeds de mogelijkheid voorzien om terug te keren naar het titelscherm met de C-knop. Vanuit het titelscherm kan je het spel afsluiten om een ander spel te laden.
	
	\begin{lstlisting}[float=!ht, language=C++, caption={\texttt{loop()}}, label={code:loop}]
	void loop(){
		if(gb.update()){
			if (gb.buttons.pressed(BTN_C)) {
				gb.titleScreen();
			}
			// GAME
		}
	}
	\end{lstlisting}
	
	
	\newpage
	\section{Gamebuino library}
	Hier geven we een overzicht van functies en nuttige variabelen in de \emph{Gamebuino} library.
	Optionele argumenten zijn schuingedrukt.
	
	\subsection{Core}
	\begin{itemize}
		\item \texttt{gb.begin()}:
		Initialiseer de Gamebuino. Moet éénmaal opgeroepen worden bij het begin van de \texttt{setup()}-functie, gevolgd door \texttt{gb.titleScreen()} en \texttt{gb.pickRandomSeed()}.
		
		\item \texttt{gb.titleScreen(\textit{F("name")}, \textit{logo})}:
		Toon het titelscherm. Moet opgeroepen worden na \texttt{gb.begin()} om het titelscherm te tonen bij het opstarten. Je moet de gebruiker ook toelaten om terug te gaan naar het titelscherm met de C-knop.
		\begin{itemize}
			\item [] \textbf{Parameters}
			\item \textit{F("name")}: Naam die wordt weergegeven op het titelscherm, vervang \textit{name} door de naam van jouw spel.
			\item \textit{logo}: Logo van jouw spel. Kan van formaat variëren tussen 8x8px en 64x30 px, en tot 64x36 px als geen naam is opgegeven.
		\end{itemize}
	
		\item \samepage \texttt{gb.update()}:
		Geeft \texttt{true} en update het display, geluid, batterij monitor, ... met een vaste frequentie (afhankelijk van de frame rate).
		\begin{itemize}
			\item [] \textbf{Return}
			\item \texttt{boolean}: \texttt{true} als voldoende tijd verstreken is sinds het laatste frame.
		\end{itemize}
	
		\item \texttt{gb.setFrameRate(fps)}:
		Pas het aantal frames per seconde aan. Bepaalt hoe vaak per seconde het programma wordt uitgevoerd (zie \texttt{gb.update()}).
		\begin{itemize}
			\item [] \textbf{Parameters}
			\item \texttt{byte} fps: Het aantal frames per seconde, standaard 20 fps.
		\end{itemize}
	
		\item \texttt{gb.pickRandomSeed()}:
		Selecteert een random seed op basis van de batterijspanning, omgevingslicht en de verstreken tijd sinds het opstarten. Moet meteen na \texttt{gb.begin()} en \texttt{gb.titleScreen()} geplaatst worden in \texttt{setup()}.
		
		\item \texttt{gb.changeGame()}:
		Flasht LOADER.HEX van de SD-kaart zodat een ander spel geladen kan worden. Terugkeren naar de loader kan ook via \texttt{gb.titleScreen()}.
		
		\item \texttt{gb.frameCount}:
		Variabele die wordt opgehoogd iedere keer dat er een nieuw frame wordt weergegeven. Aantal frames dat werd gerenderd sinds het programma begon met uitvoeren.
		
		\item \texttt{gb.getDefaultName(name)}:
		Verkrijg de gebruikersnaam die ingesteld is met SETTINGS.HEX.
		\begin{itemize}
			\item [] \textbf{Parameters}
			\item \texttt{char*} name: 10 char lange string waarin de gebruikersnaam wordt opgeslagen.
		\end{itemize}
	\end{itemize}
	
	\subsubsection{User Interface}
	\begin{itemize}
		\item \texttt{gb.menu(menu, length)}:
		Toont een menu met items om uit te kiezen.
		\begin{itemize}
			\item [] \textbf{Parameters}
			\item \texttt{char** PROGMEM} menu: de items waaruit gekozen kan worden, opgeslagen als een \texttt{PROGMEM}-array van strings.
			\item \texttt{byte} length: Aantal items in het menu.
		\end{itemize}
		\samepage
		\begin{itemize}
			\item [] \textbf{Return}
			\item \texttt{char} number: Nummer van het geselecteerde item of -1 als het menu verlaten is zonder een item te kiezen.
		\end{itemize}
	
		\item \texttt{gb.keyboard(string, length)}:
		Toon een keyboard zodat de gebruiker tekst kan invoeren.
		\begin{itemize}
			\item [] \textbf{Parameters}
			\item \texttt{char*} string: Ontvangt de tekst ingevoerd door de gebruiker.
			\item \texttt{byte} length: Maximale lengte van de tekst die de gebruiker kan ingeven.
		\end{itemize}
	
		\item \texttt{gb.popup(F("Text"), duration)}: Toon een pop-up melding aan de onderkant van het scherm voor bepaalde duur.
		\begin{itemize}
			\item [] \textbf{Parameters}
			\item \texttt{PROGMEM char*} F("Text"): Tekst die weergegeven moet worden.
			\item \texttt{byte} duration: Hoe lang wordt de pop-up weergegeven, uitgedrukt in aantal frames.
		\end{itemize}
	\end{itemize}
	
	\section{Testen}
	% emulator
	% compileren in Arduino IDE
	
	\section{Programma op Gamebuino plaatsen}
	De Gamebuino laadt bij het opstarten een bootloader, het menu waar je verschillende spelletjes kan kiezen. Om je eigen programma in deze lijst weer te geven, moet je het gecompileerde programma op de SD-kaart plaatsen. Het gecompileerde programma is een bestand met als extensie \emph{.hex}. Dit bestand moet je hernoemen zodat de naam enkel hoofdletters bevat en een maximale lengte heeft van 8 karakters voor de extensie, zoals in Figuur \ref{fig:hernoemen}.
	\begin{figure}[h]
		\centering
		\fbox{
			\parbox{8cm}{
				\centering
				Pong.hex \(\longrightarrow\) PONG.HEX \\
				SpaceInvaders.hex \(\longrightarrow\) SPACEINV.HEX
			}
		}
		\caption{\label{fig:hernoemen}Hernoemen van gecompileerde programma's}
	\end{figure}
	
	% INF encoder
	Als je wil dat jouw game met logo en beschrijving wordt weergegeven in het menu, moet je ook een \emph{.INF}-bestand maken met dezelfde naam. Dit doe je met een \textbf{INF encoder}\cite{GitHub:Rodot:InfEncoder}. Deze encoder zet die informatie om naar een bestand dat de Gamebuino-loader kan lezen. Hiervoor heb je een logo nodig van 19x18 px en je kan tot 255 slides van 84x32 px toevoegen die meer vertellen over de game. De afbeeldingen die je gebruikt voor het logo of de slides mogen enkel de kleuren wit en zwart bevatten, daarom kan je hiervoor best een monochrome bitmap gebruiken als bestandsformaat.
	
	Nu heb je twee bestanden met dezelfde naam en de extensies \emph{.HEX} en \emph{.INF}. Door deze twee bestanden op de SD-kaart te plaatsen, zal jouw eigen game nu ook in het opstartmenu verschijnen!
	
	\begin{figure}[h]
		\centering
		\fbox{\includegraphics[height=3cm]{assets/img/icon}}
		\fbox{\includegraphics[height=3cm]{assets/img/slide1}}
		\caption{\label{fig:logo_slide}Voorbeeld van een logo en slide voor Pong, \textbf{niet} op gelijke schaal}
	\end{figure}
	
	\newpage
	\bibliographystyle{plain}
	\bibliography{document}
\end{document}